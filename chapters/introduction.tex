\chapter{Introduction}

This thesis emerged from a practical problem encountered during the development
of a modern cloud storage: how can a geo-distributed storage system efficiently
detect data corruption without assuming that all nodes are always online? The answer, as explored in this work, lies in rethinking corruption detection through the use of hierarchical data structures and consensus algorithms.

\section{Motivation}

File corruption can silently compromise stored data, reduce reliability, or increase recovery costs. Traditional approaches such as checksums, replication, or RAID \cite{chen1994raid} provide partial solutions, but they also show limitations when applied to large-scale and geo-distributed environments. In particular, these methods often assume constant node availability or require full scans that become increasingly expensive as data grows.

During my internship at Cubbit\footnote{\url{https://cubbit.io}}, the first geo-distributed cloud storage provider, I studied the company's existing corruption detection system, which relies on checksum-based verification. While functional, this approach has two key weaknesses: it assumes that nodes are always online, and it requires heavy operations to verify integrity. Moreover, Cubbit stores files using Reed-Solomon codes, meaning that data is distributed into shards across multiple agents, and files are reconstructed from subsets of those shards. This introduces additional challenges, since nodes holding useful shards may be offline at any time.

Motivated by these challenges, I explored an alternative approach based on Merkle trees. Merkle trees, widely used in blockchain and distributed databases, allow efficient and hierarchical verification of large datasets: instead of re-checking entire files, integrity can be confirmed by verifying only a logarithmic number of hashes along a path in the tree. Combined with a consensus algorithm, this structure allows folder-level integrity verification even in clusters where nodes frequently join and leave.

\section{Problem Statement}
Verifying file integrity at the file level quickly becomes inefficient as the dataset grows. In Cubbit's environment, the problem is further complicated by node churn: data may be temporarily unavailable, yet the system must still guarantee correctness.

Cubbit's use of Reed-Solomon coding ensures that files can be reconstructed even when some shards are missing or corrupted. However, Reed-Solomon alone does not provide a way to verify the integrity of individual shards. This means that a reconstructed file could include corrupted data without the system being able to detect where the corruption occurred.

Merkle trees address this gap, providing an efficient mechanism to detect and localise corruption by verifying a logarithmic number of hashes along an authentication path. The key challenge lies in constructing and verifying Merkle trees in a distributed setting, where some shards are stored on agents that may be offline.

This thesis addresses the need for a corruption detection system that:
\begin{itemize}
\item avoids repeated full-file scans by verifying integrity at the folder and sub-folder level;
\item tolerates offline nodes by storing and synchronizing integrity metadata across the cluster;
\item integrates with Cubbit's existing Reed-Solomon codes, ensuring that verification is possible even during partial uploads and recoveries.
\end{itemize}

The central question is: how can a Merkle-tree-based corruption detection system, combined with Raft consensus and compatible with Reed-Solomon codes, provide efficient, scalable, and fault-tolerant integrity verification in a geo-distributed cluster?

\section{Research Objectives}

The primary objective of this thesis is to design and evaluate a corruption detection system for distributed storage clusters that is both efficient and resilient to node failures. Unlike checksum-based approaches, the proposed system leverages:  
\begin{itemize}
    \item an ad-hoc Merkle tree library optimized for folder-based hierarchies;  
    \item coordination via Raft consensus to ensure consistent integrity metadata across nodes;  
    \item integration with Cubbit's Reed-Solomon-based infrastructure to handle partial uploads and recoveries;  
    % \item evaluation of different cryptographic hash functions (SHA-256 vs BLAKE3) for performance trade-offs.  
\end{itemize}

The overarching goal is to demonstrate that an optimized, folder-oriented Merkle tree, combined with consensus, can provide a scalable, fault-tolerant, and efficient alternative to classical checksum-based corruption detection in geo-distributed clusters.


\section{Structure of the Document}

This document is organized as follows:

Chapter 2 provides the necessary background and serves as a literature review. It examines the key technologies and theoretical foundations relevant to this work, including Merkle trees, cryptographic hash functions, consensus algorithms, and the core component of the Cubbit infrastructure. For each of these elements, different solutions proposed in the literature are analysed and compared. The chapter then introduces the selected solutions, together with the rationale for their adoption, explaining why they are best suited to the corruption detection system under consideration.
    
Chapter 3 describes the architecture of the proposed corruption detection system. It explains how the components introduced in Chapter 2 are combined and how they interact within the overall design.

Chapter 4 presents the prototype implementation and the experimental evaluation. The system is tested under different scenarios, varying both the number of agents in the cluster and the number of files stored per agent. Controlled corruption is deliberately injected into the data, and the evaluation measures the elapsed time between the initiation of a corruption check and the correct identification of the corrupted file.

Chapter 5 concludes the thesis by summarising the main contributions, highlighting limitations, and outlining possible directions for future work.
