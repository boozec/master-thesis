\shipout\null
\chapter*{Sommario}
\thispagestyle{empty} 

Questa tesi propone un protocollo di verifica dell'integrità per sistemi di
storage geo-distribuiti, basato su alberi di Merkle, algoritmo di consenso Raft
e codifica Reed-Solomon. Il protocollo consente di garantire verifiche dei dati
efficienti e affidabili anche in ambienti in cui i nodi possono essere
temporaneamente offline, superando i limiti delle tradizionali scansioni di file
completi basate su checksum.

Un prototipo è stato implementato tramite una libreria Rust dedicata, capace di
generare in modo efficiente gli hash delle radici degli alberi di Merkle per
intere cartelle, integrata con Raft per assicurare la coerenza dei metadati tra
i nodi distribuiti. La sperimentazione dimostra che il protocollo verifica e
localizza in modo affidabile la corruzione dei dati in diversi scenari,
mantenendo correttezza anche in caso di disponibilità parziale del cluster. Il
tempo di verifica, però, aumenta con la dimensione del cluster a causa dei costi di coordinazione e del sovraccarico di rete.

I risultati indicano che un'architettura basata su alberi di Merkle, coordinata tramite consenso, offre una soluzione robusta e tollerante ai guasti per la verifica distribuita dell'integrità dei dati, suggerendo al contempo possibilità di ottimizzazione attraverso concorrenza, organizzazione adattiva dei file e deployment su larga scala.
